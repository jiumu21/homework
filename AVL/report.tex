\documentclass[UTF8]{ctexart}
\usepackage{geometry, CJKutf8}
\geometry{margin=1.5cm, vmargin={0pt,1cm}}
\setlength{\topmargin}{-1cm}
\setlength{\paperheight}{29.7cm}
\setlength{\textheight}{25.3cm}

% useful packages.
\usepackage{amsfonts}
\usepackage{amsmath}
\usepackage{amssymb}
\usepackage{amsthm}
\usepackage{enumerate}
\usepackage{graphicx}
\usepackage{multicol}
\usepackage{fancyhdr}
\usepackage{layout}
\usepackage{listings}
\usepackage{float, caption}

\lstset{
    basicstyle=\ttfamily, basewidth=0.5em
}

% some common command
\newcommand{\dif}{\mathrm{d}}
\newcommand{\avg}[1]{\left\langle #1 \right\rangle}
\newcommand{\difFrac}[2]{\frac{\dif #1}{\dif #2}}
\newcommand{\pdfFrac}[2]{\frac{\partial #1}{\partial #2}}
\newcommand{\OFL}{\mathrm{OFL}}
\newcommand{\UFL}{\mathrm{UFL}}
\newcommand{\fl}{\mathrm{fl}}
\newcommand{\op}{\odot}
\newcommand{\Eabs}{E_{\mathrm{abs}}}
\newcommand{\Erel}{E_{\mathrm{rel}}}

\begin{document}

\pagestyle{fancy}
\fancyhead{}
\lhead{张靖明, 3230106094}
\chead{数据结构与算法第6次作业}
\rhead{Nov.10th, 2024}

\section{remove函数的设计思路}
	
首先检查输入的根节点是否为nullptr:如果是,抛出UnderflowException异常。

然后根据要删除的值x与当前节点的值进行比较,直到找到x的位置。

如果当前节点有两个子树:调用detachMin函数(此时的detachMin函数已经更新相关节点的高度)找到右子树中的最小节点,这个最小节点将用来替换当前要删除的节点。将找到的最小节点的左子树设置为当前节点的左子树,右子树设置为当前节点的右子树,这样就保持了二叉搜索树的性质。删除当前节点(释放内存),并让t指向找到的最小节点,完成替换。

如果当前节点只有一个或没有子树:创建一个临时指针pos指向当前节点,修改指向,最后删除当前节点。

最后的balance函数会逆着搜索路径,对该路径的每个节点进行平衡,最后达到整个AVL树的平衡。
\end{document}

%%% Local Variables: 
%%% mode: latex
%%% TeX-master: t
%%% End: 
