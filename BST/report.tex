\documentclass[UTF8]{ctexart}
\usepackage{geometry, CJKutf8}
\geometry{margin=1.5cm, vmargin={0pt,1cm}}
\setlength{\topmargin}{-1cm}
\setlength{\paperheight}{29.7cm}
\setlength{\textheight}{25.3cm}

% useful packages.
\usepackage{amsfonts}
\usepackage{amsmath}
\usepackage{amssymb}
\usepackage{amsthm}
\usepackage{enumerate}
\usepackage{graphicx}
\usepackage{multicol}
\usepackage{fancyhdr}
\usepackage{layout}
\usepackage{listings}
\usepackage{float, caption}

\lstset{
    basicstyle=\ttfamily, basewidth=0.5em
}

% some common command
\newcommand{\dif}{\mathrm{d}}
\newcommand{\avg}[1]{\left\langle #1 \right\rangle}
\newcommand{\difFrac}[2]{\frac{\dif #1}{\dif #2}}
\newcommand{\pdfFrac}[2]{\frac{\partial #1}{\partial #2}}
\newcommand{\OFL}{\mathrm{OFL}}
\newcommand{\UFL}{\mathrm{UFL}}
\newcommand{\fl}{\mathrm{fl}}
\newcommand{\op}{\odot}
\newcommand{\Eabs}{E_{\mathrm{abs}}}
\newcommand{\Erel}{E_{\mathrm{rel}}}

\begin{document}

\pagestyle{fancy}
\fancyhead{}
\lhead{张靖明, 3230106094}
\chead{数据结构与算法第5次作业}
\rhead{Nov.3th, 2024}

\section{remove函数的设计思路}
	首先检查当前节点t是否为nullptr,如果是,则说明树为空,无法进行删除操作,此时抛出UnderflowException异常。 

	其次,通过递归找到要删除的节点:如果要删除的元素x小于当前节点的值t->element,则在左子树中递归调用remove函数继续查找要删除的节点,即通过不断深入左子树来找到目标节点;如果x大于当前节点的值,则在右子树中递归调用remove函数进行同样的操作。
	当找到要删除的节点时,先对该节点进行初步判断:如果当前节点t有两个子树,需要找到合适的节点来替换当前节点以完成删除操作。调用detachMin函数在当前节点的右子树中找到最小节点,并将其赋值给currentpos。这个最小节点将用来替换当前要删除的节点。将找到的最小节点currentpos的左子树设置为当前节点t的左子树,右子树设置为当前节点t的右子树。这样就确保了替换后的节点能够正确地连接到原来节点的子树。删除原来的节点t。最后将当前节点t指向替换后的节点。
	
	如果只有一个子树或没有子树时,创建一个临时指针pos指向当前节点t,
如果当前节点t的左子树为空,那么将当前节点t指向其右子树,即用右子树来替换当前节点。如果右子树为空,则将当前节点t指向其左子树,即用左子树来替换当前节点。删除临时指针pos指向的节点,完成删除操作。

\section{测试的结果与分析}
 
The initial Tree is: 

2

6

8

9

10

12

15

16

After the first removing,the current Tree is:

2

6

9

10

12

15

16

After the second removing,the current Tree is:

2

9

10

12

15

16

After the third removing,the current Tree is:

2

9

10

12

15

terminate called after throwing an instance of 'UnderflowException'

创建一个整数类型的二叉搜索树,通过一系列的insert操作插入整数节点:8、6、12、2、9、10、15、16。此时,二叉搜索树根据插入的节点值形成了特定的树结构,之后打印初始二叉树,以便观察插入节点后的树的形态。

第一次调用remove函数,删除根节点值为 8 的节点。根据二叉搜索树的删除算法,会根据节点的子树情况进行不同的处理。如果节点有两个子树,会找到右子树中的最小节点来替换被删除的节点;如果节点只有一个子树或没有子树,则直接用子树或nullptr来替换被删除的节点。再次调用printTree函数打印出删除根节点后的二叉搜索树结构,观察树的变化。
第二次调用remove函数,此时删去的节点只有一个子节点同样根据删除算法进行处理。再次调用printTree函数打印出此时的二叉搜索树结构。

第三次调用remove函数,删去叶子节点,该类节点的删除相对简单,直接将其父节点中指向该叶子节点的指针置为nullptr即可。

最后,调用makeEmpty函数清空二叉搜索树:调用remove(10)尝试删除值为 10 的节点,但此时树已为空,这个操作不会产生实际效果。

\end{document}

%%% Local Variables: 
%%% mode: latex
%%% TeX-master: t
%%% End: 
