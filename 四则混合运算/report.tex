\documentclass[UTF8]{ctexart}
\usepackage{geometry, CJKutf8}
\geometry{margin=1.5cm, vmargin={0pt,1cm}}
\setlength{\topmargin}{-1cm}
\setlength{\paperheight}{29.7cm}
\setlength{\textheight}{25.3cm}

% useful packages.
\usepackage{amsfonts}
\usepackage{amsmath}
\usepackage{amssymb}
\usepackage{amsthm}
\usepackage{enumerate}
\usepackage{graphicx}
\usepackage{multicol}
\usepackage{fancyhdr}
\usepackage{layout}
\usepackage{listings}
\usepackage{float, caption}

\lstset{
    basicstyle=\ttfamily, basewidth=0.5em
}

% some common command
\newcommand{\dif}{\mathrm{d}}
\newcommand{\avg}[1]{\left\langle #1 \right\rangle}
\newcommand{\difFrac}[2]{\frac{\dif #1}{\dif #2}}
\newcommand{\pdfFrac}[2]{\frac{\partial #1}{\partial #2}}
\newcommand{\OFL}{\mathrm{OFL}}
\newcommand{\UFL}{\mathrm{UFL}}
\newcommand{\fl}{\mathrm{fl}}
\newcommand{\op}{\odot}
\newcommand{\Eabs}{E_{\mathrm{abs}}}
\newcommand{\Erel}{E_{\mathrm{rel}}}

\begin{document}

\pagestyle{fancy}
\fancyhead{}
\lhead{张靖明, 3230106094}
\chead{数据结构与算法项目作业}
\rhead{12月18日, 2024}

\section{expression\_evaluator.h的设计思路}
	
这一部分主要包括(一)运算符优先级定义函数;(二)三类判断函数:分别判断字符是否代表数字、运算符以及是否为合法字符;(三)两数间的四则计算;(四)判断中缀表达式是否合法以及将其转化为后缀进行计算;(五)对转化后的后缀表达式的计算函数。

这其中,最为重要的是check\_and\_convert函数的设计。首先获取输入字符串 s 的长度,并初始化一些变量,包括 point(用于标记是否已经出现过小数点)、data(用于标记是否出现过有效数据字符)和 mark(用于标记是否出现过运算符)。如果字符串为空,返回 false。接着对第一个字符进行处理,考虑其可能情况,逐一分析。之后对每一个字符进行排查,考虑数字字符、运算符、小数点、以及两种括号的情况,实现了将一个可能包含括号、小数点和运算符的字符串转换为后缀表达式的功能,同时进行了严格的格式检查以确保输入的合法性。

该项目四则运算要求至少出现一次运算符(“(2)”判定非法),可以排除运算符连续使用、除数为0(0.0、0.00等视为0看待)、括号不匹配、小数点使用错误、括号与数字的不恰当连用等情况。注意:“(1+4)”、“((1×2))”等视为合法表达。

\section{main.cpp的设计思路}

这一部分my\_test给出25个例子,包含括号不匹配、出现非法字符、运算符连续使用、运算符出现在首位或末位、括号间无运算符、括号与数字间无运算符、除数为0、小数点使用不当等。

同时包含input,对使用者输入的字符串进行判断。

\section{The Result}

ILLEGAL

ILLEGAL

ILLEGAL

ILLEGAL

ILLEGAL

ILLEGAL

ILLEGAL

ILLEGAL

ILLEGAL

ILLEGAL

ILLEGAL

ILLEGAL

ILLEGAL

ILLEGAL

ILLEGAL

ILLEGAL

ILLEGAL

The result is : 29.56

The result is : 11.5

The result is : 2

The result is : 0.2

The result is : 6.5

The result is : 3.4

The result is : 0.922226

The result is : -5

Input : (此处支持输入你的表达式)
\end{document}

%%% Local Variables: 
%%% mode: latex
%%% TeX-master: t
%%% End: 
