\documentclass[UTF8]{ctexart}
\usepackage{geometry, CJKutf8}
\geometry{margin=1.5cm, vmargin={0pt,1cm}}
\setlength{\topmargin}{-1cm}
\setlength{\paperheight}{29.7cm}
\setlength{\textheight}{25.3cm}

% useful packages.
\usepackage{amsfonts}
\usepackage{amsmath}
\usepackage{amssymb}
\usepackage{amsthm}
\usepackage{enumerate}
\usepackage{graphicx}
\usepackage{multicol}
\usepackage{fancyhdr}
\usepackage{layout}
\usepackage{listings}
\usepackage{float, caption}

\lstset{
    basicstyle=\ttfamily, basewidth=0.5em
}

% some common command
\newcommand{\dif}{\mathrm{d}}
\newcommand{\avg}[1]{\left\langle #1 \right\rangle}
\newcommand{\difFrac}[2]{\frac{\dif #1}{\dif #2}}
\newcommand{\pdfFrac}[2]{\frac{\partial #1}{\partial #2}}
\newcommand{\OFL}{\mathrm{OFL}}
\newcommand{\UFL}{\mathrm{UFL}}
\newcommand{\fl}{\mathrm{fl}}
\newcommand{\op}{\odot}
\newcommand{\Eabs}{E_{\mathrm{abs}}}
\newcommand{\Erel}{E_{\mathrm{rel}}}

\begin{document}

\pagestyle{fancy}
\fancyhead{}
\lhead{张靖明, 3230106094}
\chead{数据结构与算法第7次作业}
\rhead{Nov.30th, 2024}

\section{HeapSort.h的设计思路}
	
首先,利用make\_heap函数将a转化成一个max堆,通过pop\_heap函数将最大值移动到最后一个位置,在此循坏中,重新调整容器,使其仍然是一个堆,但范围缩小了一个元素,以此类推,最终使得序列排序完成。

\section{test.cpp设计思路}

首先对于check函数,利用循坏结构,对于接收到的数组从下标i=1开始逐次与前一个进行比较,如果小于前一个,说明不是升序排列,返回false;如果后一个元素都不小于前一个元素,则返回true。

接着涉及四个函数分别为Random()、Ordered()、ReverseOrdered()、Repeated()构建包含1000000个元素的随机序列、有序序列、逆序序列和有重复元素的序列。

通过循环结构进行四次不同序列的测试,在每一次测试中,将上述对应的函数返回值赋给sequence,并对该序列进行拷贝,得到Copysequence。然后对利用HeapSort函数对sequence继续排序,并记录所花费的时间,对于排序好的sequence,check一下,如果检验正确,输出ture,否则输出false,接着输出所用时间;对于Copysequence,首先检测一下是否为max堆,如果不是,利用make\_heap()使其成为一个max堆,之后用sort\_heap()进行排序并且记录花费时间,之后操作同上述sequence序列。

\section{Result}

随机序列通过HeapSort排序用时 : 548ms 排序正确

随机序列通过sort\_heap排序用时 : 536ms 排序正确

有序序列通过HeapSort排序用时 : 276ms 排序正确

有序序列通过sort\_heap排序用时 : 248ms 排序正确

逆序序列通过HeapSort排序用时 : 255ms 排序正确

逆序序列通过sort\_heap排序用时 : 256ms 排序正确

重复序列通过HeapSort排序用时 : 464ms 排序正确

重复序列通过sort\_heap排序用时 : 435ms 排序正确

\end{document}

%%% Local Variables: 
%%% mode: latex
%%% TeX-master: t
%%% End: 
